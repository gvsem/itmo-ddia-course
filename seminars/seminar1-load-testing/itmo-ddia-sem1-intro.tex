\documentclass{beamer}
\usetheme{Madrid}
\usepackage{minted}
% sudo tlmgr install minted
\usepackage[utf8]{inputenc}
\usepackage[T2A]{fontenc}
\usepackage[russian,english]{babel}

% Footline: show current section (left) and frame numbers (right)
\setbeamertemplate{footline}{%
  \leavevmode%
  \hbox{%
    \begin{beamercolorbox}[wd=.78\paperwidth,ht=2.5ex,dp=1ex,left]{author in head/foot}%
      \hspace*{1em}\usebeamerfont{footline}\insertsectionhead%
    \end{beamercolorbox}%
    \begin{beamercolorbox}[wd=.22\paperwidth,ht=2.5ex,dp=1ex,right]{date in head/foot}%
      \usebeamerfont{footline}\insertframenumber{} / \inserttotalframenumber\hspace*{1em}%
    \end{beamercolorbox}%
  }%
}

\newcommand{\parpause}[1]{\only<+->{#1\par}}

\AtBeginSection[]
{
  \begin{frame}
    \frametitle{Содержание}
    \tableofcontents[currentsection]
  \end{frame}
}

\title{Семинар 1. Мониторинг и нагрузочное тестирование}
\subtitle{Принципы построения высоконагруженных систем}
\author{Георгий Семенов}
\institute{Институт прикладных компьютерных наук \\ Университет ИТМО}
\date{осень 2025}

\begin{document}

\frame{\titlepage}

\begin{frame}
\frametitle{Домашние задания и оценивание}

\begin{itemize}
	\item Задание выдается на $2$ недели с двумя дедлайнами:
	      \begin{itemize}
	      	\item \textbf{Мягкий дедлайн} – в рамках этого периода можно заранее отправить решение
	      	      и получить обратную связь, чтобы исправить замечания до наступления жесткого дедлайна.
	      	\item \textbf{Жесткий дедлайн} – после этого новые посылки работы не принимаются, сдать работу больше нельзя.
	      \end{itemize}
	\item Планируется выдать $4$ домашних задания, каждое из которых стоит $\pm 12$ баллов. Правила выставления оценки за курс следующие:
	      \begin{itemize}
	      	\item \textbf{A -- <<5>>} – $\geq 90\%$ от общей суммы обязательных баллов (предв. $\geq 43.2$)
	      	\item \textbf{B/C -- <<4>>} – $\geq 75\%$ от общей суммы обязательных баллов (предв. $\geq 36$)
	      	\item \textbf{D/E -- <<3>>} – $\geq 60\%$ от общей суммы обязательных баллов (предв. $\geq 28.8$)
	      	\item \textbf{F -- <<2>>} – $< 60\%$ от общей суммы обязательных баллов (предв. $< 28.8$)
	      \end{itemize}
\end{itemize}

\end{frame}

\begin{frame}
\frametitle{Как отправлять домашние задания?}

\begin{itemize}
	\item Выполненное задание рекомендуется оформить в \LaTeX (например, в Overleaf) и выслать файлом с именем вида \texttt{DDIA25-HW1-IvanIvanov.pdf}
	      на почту \texttt{georgii.v.semenov@mail.ru}.
	      \begin{itemize}
	      	\item Пожалуйста, оформляйте ответы на задания в блоке \texttt{Решение}.
	      \end{itemize}
\end{itemize}

\end{frame}

\section{Мониторинг}

% https://docs.docker.com/compose/install/
% 

\begin{frame}
  \frametitle{Мотивация}
  \begin{itemize}
    \parpause{\item \textbf{Backend} – сеть геораспределенных черных ящиков}
    \parpause{\item Ящики расположены в разных дата-центрах, у них разные нагрузки и поведение}
    \parpause{\item Хотим \textbf{observability} – представление о том, как дела у ящиков}
    \parpause{\item С ней можем обеспечивать и оценивать \textbf{reliability}}
  \end{itemize}
\end{frame}

\begin{frame}
  \frametitle{SRE – Software Reliability Engineering}

  \begin{center}
    \includegraphics[width=0.8\linewidth,keepaspectratio]{images/sre_books.png}
  \end{center}

  \small
  \begin{itemize}
    \item SRE\footnote{Ссылки на книги: \href{https://sre.google/books/}{https://sre.google/books}} как дисциплина определяет подходы к \textbf{reliability} и \textbf{observability} систем.
    \item SLA (Service Level Agreement) – гарантии и последствия их нарушения
    \item SLO (Service Level Objective) – метрики и целевые требования к ним
    \item SLI (Service Level Indicator) – измеряемые метрики
  \end{itemize}
\end{frame}

\begin{frame}
  \frametitle{Что такое мониторинг?}
  \begin{itemize}
    \parpause{\item Собираем показатели с каждого пода каждого узла системы}
    \parpause{\item Строим временные ряды по этим показателям}
    \parpause{\item Оцениваем показатели SLI, стремимся обеспечить SLO, соблюдаем SLA}
    \parpause{\item Настраиваем \textbf{alerts}, \textbf{tickets}, \textbf{logging} и реагируем на инциденты дежурной сменой}
  \end{itemize}
\end{frame}

\begin{frame}
  \frametitle{Метрики системы: ресурсы}
  \begin{itemize}
    \item \textbf{CPU usage}: утилизация, параллелизм, троттлинг...
    \item \textbf{Memory usage}: утилизация по типам памяти, подкачка...
    \item \textbf{Disk I/O}: объем дисков, рейт чтения/записи, буферы, задержки...
    \item \textbf{Network I/O}: рейт чтения/записи, задержки пакетов TCP/UDP...
    \item \textbf{GC, Heap Memory} – параметры для сред Java, Go и т.п.
    \item \textbf{Queues} – парааметры очередей заданий
    \item \textbf{Connections} – количество открытых соединений
    \item \textbf{Threads} – количество активных процессов, потоков
  \end{itemize}
\end{frame}

\begin{frame}
  \frametitle{Метрики системы: запросы}
  \begin{itemize}
    \item \textbf{QPS}, или Throughput (requests per second)
    \item \textbf{Timing}, или Latency (response time)
    \item \textbf{Error rate} – RPS ошибок
    \item \textbf{Uptime}
    \item \textbf{Downtime}
    \item \textbf{Availability SLO}: Error Budget, Uptime %
  \end{itemize}
\end{frame}

\begin{frame}
  \frametitle{Метрики системы: бизнес-метрики}
  \begin{itemize}
    \item Активность пользователей: Time Spent, Active Users (MAU/WAU/DAU/PAU)
    \item Конверсия: конвертируемость пользователей вдоль CJM
    \item Транзакции: успешно завершенные пользовательские сценарии
    \item Доход: выручка (ARPU, ARPPU), юниты
  \end{itemize}
\end{frame}

\begin{frame}
  \frametitle{Золотые сигналы}
  % https://www.dynatrace.com/knowledge-base/golden-signals/
  
  Базовый минимум для мониторинга системы:

  \begin{itemize}
    \item \textbf{Latency} – семейство измерений времени отклика
    \item \textbf{Traffic} – семейство измерений входящего трафика
    \item \textbf{Errors} – семейство измерений ошибок
    \item \textbf{Saturation} – семейство измерений загрузки системы
  \end{itemize}
\end{frame}

\section{Временные ряды}

\begin{frame}[fragile]
\frametitle{Понятие метрики}
\begin{itemize}
\item Метрика — набор: \texttt{timestamp}, \texttt{value}, \texttt{labels}.
\item У каждой метрики есть служебный label \texttt{\_\_name\_\_}
\item Пример метрики:
\begin{minted}[escapeinside=||]{promql}
  node_manager_node_up{node="n1"}
\end{minted}

\item То же самое, но без синтаксического сахара:
\begin{minted}[escapeinside=||]{promql}
  {__name__="node_manager_node_up", node="n1"} 
\end{minted}
\end{itemize}

\end{frame}

\begin{frame}
\frametitle{Типы метрик (в Prometheus)}
\begin{itemize}
\item \textbf{Counter} — монотонно растёт (например, суммарное количество запросов).
\item \textbf{Gauge} — текущее значение, оно может как увеличиваться, так и уменьшаться.
\item \textbf{Histogram} — распределение величины по \textit{buckets} (счетчики по бакетам).
\item \textbf{Summary} — похож на histogram, даёт квантильные оценки, но с искажениями при агрегации.
\end{itemize}
\end{frame}

\begin{frame}
\frametitle{PromQL: типы значений}
\begin{itemize}
\item \textbf{Instant Vector} — множество временных рядов в конкретный момент (набор пар \{labels, value\}).
\item \textbf{Range Vector} — для каждой временной серии набор samples за интервал (для функций \texttt{over\_time}, \texttt{rate}, \texttt{increase} и т.д.).
\item \textbf{Scalar} — единичное числовое значение (результат вычисления). Иногда используется в alert expressions.
\end{itemize}
\end{frame}

\section{Prometheus и PromQL}

\begin{frame}
  \frametitle{СУБД для временных рядов}

  \begin{itemize}
    \item Оптимизированы для данных вида \textit{timestamp → value}.
    \item Высокая скорость записи (до миллионов точек в секунду).
    \item Эффективная компрессия (TSDB, columnar storage).
    \item Поддержка downsampling (сужение), retention (TTL), агрегирования.
    \item Пример: Prometheus TSDB, VictoriaMetrics, M3, InfluxDB.
  \end{itemize}

\end{frame}

\begin{frame}
  \frametitle{Prometheus}

  \begin{itemize}
    \item Open-source TSDB и система мониторинга.
    \item Pull-модель: опрашивает endpoints (\texttt{/metrics}).
    \item Язык запросов: PromQL.
    \item \textbf{Job} → \textbf{Instance} → \textbf{Metric} → \textbf{Labels}.
    \item Хранение локально, без распределённости (по дизайну).
    \item Экспорт данных в long-term: Thanos / VictoriaMetrics.
  \end{itemize}

\end{frame}

\begin{frame}
\frametitle{\href{https://megamorf.gitlab.io/cheat-sheets/prometheus/}{Примеры} Node Exporter}
\begin{itemize}
\item CPU: \texttt{node\_cpu\_seconds\_total\{mode="user"\}} 
\item Memory: \texttt{node\_memory\_MemAvailable\_bytes}, \texttt{node\_memory\_MemFree\_bytes}.
\item Disk: \texttt{node\_filesystem\_avail\_bytes}.
\item Network: \texttt{node\_network\_receive\_bytes\_total}.
\item Load (avg for mins): \texttt{node\_load1}, \texttt{node\_load5}, \texttt{node\_load15}.
\item Filesystem saturation: \texttt{node\_disk\_io\_time\_seconds\_total}.
\end{itemize}
\vspace{1em}
Пример запроса: \texttt{rate(node\_network\_receive\_bytes\_total[5m])}
\end{frame}

\begin{frame}
  \frametitle{VictoriaMetrics}

  \href{https://last9.io/blog/prometheus-vs-victoriametrics/}{Link}

  \begin{itemize}
    \item Совместима с Prometheus.
    \item Язык запросов: MetricsQL (обратно совместим с PromQL).
    \item Более эффективное хранение: до 7–10x компрессия.
    \item Дает горизонтальное масштабирование (кластерная версия).
    \item Может заменять Prometheus или быть хранилищем “за ним”.
  \end{itemize}

\end{frame}

\begin{frame}
  \frametitle{Grafana}

  \begin{itemize}
    \item Универсальная система визуализации данных.
    \item Поддерживает множество источников: Prometheus, VM, PostgreSQL, ElasticSearch, CSV...
    \item Дашборды: графики, таблицы, heatmap, alert panels.
    \item Поддержка переменных, шаблонов, drilldown.
    \item Экспорт/импорт дашбордов JSON.
  \end{itemize}

\end{frame}

\begin{frame}[fragile]
  \frametitle{Алертинг в Prometheus}

  \begin{itemize}
    \item Два компонента:
      \begin{itemize}
        \item \textbf{Prometheus alerting rules} — проверка условий (\texttt{alert:}).
        \item \textbf{Alertmanager} — маршрутизация уведомлений.
      \end{itemize}
    \item Каналы: Email, Slack, PagerDuty, Telegram, Webhook.
    \item Поддерживает инциденты, группировку, подавление, тишину (silences).
    \item Пример правила:
  \end{itemize}

\footnotesize
\begin{minted}[escapeinside=||]{yaml}
  alert: HighCpuUsage
  expr: avg(rate(node_cpu_seconds_total{mode!="idle"}[5m])) > 0.8
  for: 10m
  labels: { severity="warning" }
  annotations: {
    summary="High CPU usage on {{ $labels.instance }}"
  }
\end{minted}

\end{frame}

\section{Демонстрация: Grafana}

\begin{frame}
  \frametitle{Куда посмотреть?}

  \begin{itemize}
    \item \texttt{grafana/docker-compose.yml}
    \item \texttt{grafana/prometheus.yml}
    \item \texttt{docker-compose up}
    \item \href{http://localhost:3000}{http://localhost:3000}
    \item \href{http://localhost:3000/connections/datasources/new}{http://localhost:3000/connections/datasources/new}
    \item \href{http://victoriametrics:8428}{http://victoriametrics:8428}
    \item \href{http://localhost:8428/targets}{http://localhost:8428/targets}
  \end{itemize}
\end{frame}

\section{Нагрузочное тестирование}

\begin{frame}
  \frametitle{Мотивация}
  \begin{itemize}
    \item Обеспечение стабильной работы системы под высокой нагрузкой
    \item Выявление узких мест производительности до выхода в прод
    \item Оценка поведения при пиковых значениях запросов
    \item Проверка корректности масштабирования
    \item Подтверждение соблюдения SLA/SLO
    \item Предотвращение деградаций и простоев
  \end{itemize}
\end{frame}

\begin{frame}
  \frametitle{Некоторые идеи}
  \begin{itemize}
    \item \textbf{Load Testing} — проверка работы при ожидаемой нагрузке
    \item \textbf{Stress Testing} — определение предельных возможностей
    \item \textbf{Spike Testing} — реакция на резкие скачки трафика
  \end{itemize}
\end{frame}

\section{Демонстрация: wrk}

\begin{frame}
  \frametitle{Подготовка}
  \begin{itemize}
    \item На MacOS: \texttt{brew install wrk}
    \item На Ubuntu: \texttt{sudo apt-get install wrk}
    \item \texttt{wrk -t4 -c50 -d20s http://localhost:8081/}
  \end{itemize}
\end{frame}

\section{Итоги}
\begin{frame}
  \frametitle{Итоги}
  \begin{itemize}
    \item Рассмотрели мониторинг и нагрузочное тестирование
    \item Рассмотрели примеры в Grafana и с wrk
    \item Выдано первое домашнее задание
  \end{itemize}
\end{frame}

\end{document}