\documentclass[12pt]{article}
\usepackage[dvipsnames]{xcolor}
\usepackage{minted}
% sudo tlmgr install minted
\usepackage{hyperref}
\usepackage[margin=1in]{geometry} 
\usepackage{amsmath,amsthm,amssymb}
\usepackage{graphicx}
\usepackage[utf8]{inputenc}
\usepackage[T2A]{fontenc}
\usepackage[russian,english]{babel}

\hypersetup{
    colorlinks=true,
    linkcolor=blue,
    filecolor=magenta,      
    urlcolor=blue,
    pdfpagemode=FullScreen,
}

\newenvironment{problem}{
    \par\textbf{Задание: }\ignorespaces
}

\newenvironment{solution}
{
  \par\textbf{Решение:}\par\bigbreak
  \begingroup
}
{
  \par\bigbreak\hfill$\square$%
  \par\endgroup
}

\begin{document}
 
\title{
    \large Принципы построения высоконагруженных систем \\
    \normalsize Институт прикладных компьютерных наук ИТМО
    \bigbreak 
    \LARGE
    Домашнее задание 2. \\ Масштабируемость и балансировка нагрузки
    \bigbreak \normalsize
    Георгий Семенов \\
    \texttt{georgii.v.semenov@mail.ru} \\
    Мягкий дедлайн: Сб, 06.12.2025, 23:59 МСК \\
    Жесткий дедлайн: Сб, 13.12.2025, 23:59 МСК
}
\author{
    \color{red}{Имя Фамилия} \\
    \normalsize
    DDIA25-HW2-{\color{red}NameSurname}.pdf
}

\maketitle

\noindent\rule{\textwidth}{1pt} \bigbreak

\break

\section{Проектирование высоконагруженных систем}{}

\begin{flushleft}
	В ходе выполнения упражнений в этом домашнем задании вам предстоит помочь корпорации \textit{<<Репликон>>}
	разобраться с вопросами масштабирования и балансировки.
\end{flushleft}

\subsection{Масштабирование}

\begin{flushleft}
Корпорация «Репликон» запускает свой видеохостинг и решила подготовить
серверное оборудование для зон облака, в которых планируется разместить его сервисы CDN.

В рамках этого процесса Cloud-команде требуется подобрать характеристики для <<стойки мечты>>.
\end{flushleft}

\begin{flushleft}
Стажер предложил использовать для всех новых стоек
одинаковые машины с максимально возможными характеристиками CPU, RAM, Disk и Network.
\end{flushleft}

\begin{flushleft}
Эта идея не очень понравилась опытным инженерам \textit{Репликона}, которые на это возразили, что
дорогие вертикальные улучшения серверов не всегда приводят к пропорциональному росту производительности
из-за аппаратных ограничений, но совершенно не пояснили, о каких именно ограничениях идет речь.
\end{flushleft} 

\begin{flushleft}
Также, инженеры предложили не использовать отдельные серверные стойки для видеохостинга, а разместить CDN-сервисы, просто горизонтально масштабируя уже существующие сервера облака \textit{Репликона}.
\end{flushleft}

\begin{problem}
	\begin{enumerate}
	\item (0.75 балл)
	Помогите стажеру понять – какие аппаратные ограничения имеют в виду опытные инженеры? \textit{(Указание: рассмотрите, как процессор, память, диск и сеть
	с точки зрения материнской платы взаимодействуют друг с другом; приведите $\geq 3$ ограничения и поясните их)}
	\item (0.75 балл)
	Почему при горизонтальном масштабировании на всех машинах стараются использовать одни и те же hardware-характеристики (что такое vCPU)?
	\end{enumerate}
\end{problem}

\begin{solution}
	
\end{solution}

\break
\subsection{Тонкости CAP-теоремы}

 \catcode`\%=12

\begin{flushleft}
Ведущий разработчик корпорации \textit{Репликон} является огромным поклонником CAP-теоремы\footnote{
	CAP-теорема (Brewer, 2000): в распределённых системах невозможно одновременно обеспечить
	все три свойства: согласованность (consistency), доступность (availability) и устойчивость к разделению (partition tolerance).
} и исследователя в области распределенных систем Мартина Клеппмана.
В очередной раз перелистывая любимого <<кабанчика>>\footnote{
	<<Кабанчик>> – это книга Мартина Клеппмана <<Высоконагруженные приложения: программирование, масштабирование, поддержка>>. Для работы над заданием можно воспользоваться текстом на
	\href{https://unidel.edu.ng/focelibrary/books/Designing%20Data-Intensive%20Applications%20The%20Big%20Ideas%20Behind%20Reliable,%20Scalable,%20and%20Maintainable%20Systems%20by%20Martin%20Kleppmann%20(z-lib.org).pdf}{русском}
	или \href{https://ftp.zhirov.kz/books/IT/Алгоритмы/Высоконагруженные%20приложения.%20Программирование%2C%20масштабирование%2C%20поддержка%20%28Мартин%20Клеппман%29.pdf}{английском} языке
}, ведущий разработчик неожиданно для себя обнаружил в книге
своего кумира <<махровый популизм>> – Мартин назвал CAP-теорему <<бесполезной>>!
\end{flushleft}

\begin{flushleft}
Ведущий разработчик был в ярости! Такого предательства он не мог простить. Он решил написать гневное письмо
Мартину Клеппману, в котором изложил бы все свои аргументы в пользу важности CAP-теоремы.
Но перед этим он решил подробнее изучить вопрос и изучить, почему же исследователь так считает.
\end{flushleft}

\begin{problem}
	\begin{enumerate}
	\item (1.5 балл)
	<<Докажите>> (или покажите) CAP-теорему, показывая несовместность каждой двойки свойств с оставшимся третьим свойством.
	Приведите по примеру соответствующих систем и соответствующий тип consistency.
	\item (0.75 балл)
	Приведите $\geq 3$ критических аргумента Мартина Клеппмана о CAP-теореме из книги (раздел <<Теорема CAP>>).
	\item (0.75 балл)
	Наряду с CAP-теоремой существует также PACELC-теорема. Сформулируйте PACELC-теорему и объясните,
	как она расширяет CAP-теорему.
	\item (2 балл)
	(*) Свойство SEC (strong eventual consistency) считается <<разрешением>> CAP-теоремы.
	Объясните, что это за свойство и почему рассматривая набор свойств \{SEC, A, P\} мы избегаем противоречия CAP-теоремы.
	\item (3 балла)
	(*) В статье <<A Critique of the CAP Theorem>> Мартин Клеппман более подробно критикует классическую формулировку CAP-теоремы,
	а также неявно предлагает альтернативную формулировку на основе <<delay-sensitivity framework>>. Сформулируйте ее.
	\end{enumerate}
\end{problem}

\begin{solution}
	
\end{solution}

\break
\subsection{Модели репликации}

Для хранения пользовательских данных в <<Репликоне>> решили воспользоваться
платформой хранения больших данных YTsaurus, а точнее – ее механизмом
\href{https://ytsaurus.tech/docs/ru/user-guide/dynamic-tables/replicated-dynamic-tables}{реплицированных динамических таблиц},
которые представляют собой key-value хранилища в реляционном стиле.

\begin{flushleft}
Если коротко, применение \textbf{модифицирующих операций} сводится к добавлению соответствующей записи в очередь операций на master-кластере.
Затем каждое изменение из лога последовательно применяется на всех slave-кластерах синхронно и/или асинхронно (это может приводить к рассогласованию данных
между master- и slave-кластерами в течение некоторого времени). Пока операция модификации не применена
на всех slave-кластерах, она не очищается из очереди на master-кластере. \textbf{Операции чтения} производятся на slave-кластерах.
\end{flushleft}

\begin{problem}
	\begin{enumerate}
	\item (1.25 балл)
	\begin{itemize}
	\item Сформулируйте общее и различия между понятиями \textit{федерация} и \textit{шардирование} в контексте баз данных.
	\item Как эти подходы помогают масштабировать системы хранения данных?
	\item Что делать, если мы уперлись в пределы вертикального масштабирования внутри одной СУБД (при федерации) / партиции (при шардировании)?
	\end{itemize}
	\item (1.25 балла)
	\begin{itemize}
	\item Опишите с точки зрения CAP-теоремы, к какому классу систем можно отнести реплицированные динамические таблицы в YTsaurus.
	\item Предположим, что один из slave-кластеров отказал. Почему в этот момент система все еще может считаться доступной?
	\item Какой тип согласованности (consistency) обеспечивается реплицированными динамическими таблицами? Это модель ACID или BASE?
	\end{itemize}
	\item (2 балла)
	(*) Утверждается, что CRDT-типы	 \footnote{
		Хабр: \href{https://habr.com/ru/companies/vk/articles/934682/}{Наивное введение в CRDT-типы}
	} способны решить задачу master-master репликации.
	Объясните, что это за типы данных и почему им не требуется механизм консенсуса для обеспечения согласованности.
	\end{enumerate}
\end{problem}

\begin{solution}
	
\end{solution}

\break
\subsection{Балансировка нагрузки }

<<Дореплицировались!>> – подумал лид Конвергентий Смоуктуновский одной из команд <<Репликона>>, когда его поисковая система
картинок с котиками <<Котоморфизм>> сложилась карточным домиком под напором пользователей.

\begin{flushleft}
CTO <<Репликона>> не остался в долгу и поручил Конвергентию разобраться с проблемой, хотя
и не очень обрадовался тому, что лид в его подчинении не догадался заранее позаботиться
о масштабировании и балансировке нагрузки в критически важном сервисе.
\end{flushleft}

\begin{flushleft}
Чтобы быстро исправить техдолг перед внедрением полноценного кэша на основе Redis, Конвергентий решил выполнить простые шаги:
\begin{enumerate}
\item Переехать из одной зоны в три геораспределенные зоны с балансировкой нагрузки между ними
	  с помощью \href{https://nginx.org/en/docs/http/ngx_http_proxy_module.html}{\textbf{nginx}}, чтобы <<Котоморфизм>> спокойно переживал отказ одной из зон.
\item Добавить \textbf{fallback} на уровне балансировщика, чтобы отображать типовые картинки с котиками,
      если <<Котоморфизм>> снова начнет <<мяукать>> под нагрузкой.
\end{enumerate}
\end{flushleft}

\begin{flushleft}
Помогите Конвергентию выполнить поставленные задачи, чтобы <<Котоморфизм>> снова начал радовать пользователей.
В папке \texttt{homeworks/cotomorphism} находится тестовый стенд из нескольких узлов <<Котоморфизма>>,
которые возвращают различные картинки с котиками. С помощью эндпоинтов \texttt{/degrade/on} и \texttt{/degrade/off}
можно включать и выключать <<режим деградации>>, при котором сервер начинает возвращать ошибку 503 вместо картинки.
\end{flushleft}

\begin{flushleft}
В вашем распоряжении baseline-конфигурация \texttt{default.conf}, которая балансирует нагрузку между тремя узлами
<<в разных зонах>> <<Котоморфизма>> с помощью round-robin.
\end{flushleft}

\begin{minted}{nginx}
upstream app_backend {
    server cat1:80;
    server cat2:80;
    server cat3:80;
}

upstream app_fallback {
    server cat_cached:80;
}

server {
    listen 80;
    server_name _;

    location / {
        proxy_pass http://app_backend;
    }

    location @fallback {
        proxy_pass http://app_fallback;
    }
}
\end{minted}

\begin{problem}
	\begin{enumerate}
	\item (0.75 балл) Сделайте так, чтобы при возникновении ошибки 503 от одного из узлов без каких-либо дальнейших попыток
	балансировщик переадресовывал запрос на upstream \texttt{app\_fallback}.
	\item (0.75 балл) Настройте балансировщик так, чтобы он пытался повторить запрос на другой узел
	при возникновении ошибок 503. Если \texttt{cat1}, \texttt{cat2}, \texttt{cat3} лежат, то необходимо спроксировать
	запрос на upstream \texttt{app\_fallback}.
	\item (0.75 балл) Добавьте кэширование запроса картинок на уровне балансировщика с временем жизни кэша 5 секунд.
	\item (0.75 балл) Добавьте \href{https://nginx.org/en/docs/http/ngx_http_limit_req_module.html}{rate limiting} на уровне балансировщика: не более 1 запроса в секунду от одного клиента.
			Убедитесь, что при превышении лимита вы падаете на \texttt{app\_fallback}.
	\end{enumerate}
\end{problem}

\textit{Указание: оформите ваше решение как минимальный набор команд, которые надо дописать в baseline-конфигурацию. Воспользуйтесь блоком minted.}.
\bigbreak

\begin{solution}

\end{solution}

\break
\subsection{Кэширование}

\begin{flushleft}
Уверенный мидл разработчик Шардимир в свободное время от работы в <<Репликоне>> играет в Minecraft и больше всего на свете любит
мод \href{https://computercraft.info/wiki/Tutorials}{ComputerCraft}, который добавляет в игру компьютеры, которые 
работают на операционной системе и поддерживают программы в виде скриптов на языке Lua.
Единственное, что огорчает Шардимира в этом моде, – это необходимость писать на совершенно бесполезном в жизни языке Lua.
\end{flushleft}

\begin{flushleft}
Шардимир получил задание от своего тимлида Конвергентия – реализовать кэш для сервиса <<Котоморфизм>> на основе
Redis. Какое же было удивление Шардимира, когда он осознал, что для этого ему придется написать пару Lua-скриптов для Redis!
\end{flushleft}

\begin{minted}{lua}
return 'Hello World'
\end{minted}

\begin{minted}{text}
EVAL "return 'Hello World'" 0
\end{minted}

\begin{flushleft}
Помогите Шардимиру реализовать \textbf{cache-aside} кэширование запросов к сервису <<Котоморфизм>> с помощью \href{https://redis.io/docs/latest/develop/programmability/eval-intro/}{Lua-скриптов для Redis}.
Поднять консоль можно с помощью папки \texttt{homeworks/redis}.
\end{flushleft}

\begin{problem}
	\begin{enumerate}
	\item (1 балл) Напишите два скрипта: один принимает в качестве аргументов поисковый запрос (строка),
	TTL (в секундах) и путь к файлу (строка); скрипт должен сохранить в Redis значение из файла по ключу поискового запроса.
	Второй должен принимать в качестве аргумента поисковый запрос (строка) и возвращать значение по этому ключу из Redis.
	Вам могут пригодиться команды \texttt{EVAL}, \texttt{GET}, \texttt{SETEX}.
	\item (1 балл) Реализуйте rate limiter для пользователя как соответствующие Lua скрипты для Redis. Вам могут пригодиться команды \texttt{INCR}, \texttt{EXPIRE}.
	\end{enumerate}
\end{problem}

\begin{solution}

\end{solution}
 
\end{document}